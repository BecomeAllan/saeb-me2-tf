%----------------------------------------------------------------------------------------------------------------
% File : exemplo.tex
%----------------------------------------------------------------------------------------------------------------

% ---
% Este capítulo, utilizado por diferentes exemplos do abnTeX2, ilustra o uso de
% comandos do abnTeX2 e de LaTeX.


\chapter{Metodologia}
% Avaliação
Este documento visa analisar o aprendizado de alunos do brasileiros no ano de 2017,
e como fatores sociais podem influenciar nesse aprendizado. Para a análise, serão utilizados dados
do Sistema de Avaliação da Educação Básica (SAEB) de 2017 divulgado pelo \citeonline{Saeb2017a)}.

A pesquisa é feita por meio de testes e questionários aplicados bienalmente ao na rede pública
e numa amostra da rede privada. A amostra é amostra composta pelo 5º e 9º anos do ensino fundamental, e também
pelo 3º ano do ensino médio. Além do desempenho em matemática e língua portuguesa, por meio da prova,
são coletadas informações a respeito dos alunos, tais como:
    
    \begin{itemize}
        \item Dependência administrativa da escola (se é federal, estadual, municipal ou privada);
        \item Grau de escolaridade dos pais; 
        \item Taxas de reprovação;
        \item Tempo diário com afazeres domésticos;
        \item Tempo diário de uso de telas (tais como computadores, celulares e televisões);
        \item Se o estudante trabalha;
        \item Quais as suas perspectivas;
        \item Acesso a computadores;
        \item Raça/cor.
    \end{itemize}


% Amostra
A análise parte de uma amostra aleatória simples de 5.271 alunos do 9º desse banco de dados.

As variáveis resposta serão a soma das notas em língua portuguesa e matemática e o tempo de afazeres domésticos realizados por dia por cada aluno.

% Relacao com notas
Como variáveis explicativas para a soma das notas, o estudo relaciona a raça/cor, a escolaridade da mãe e as localizações das escolas,
e avalia o quanto essas variáveis influenciam no desempenho na Prova Brasil. Para avaliar estas relações, um estudo prévio foi realizado
com amostras de tamanhos 30, 50 e 100 dos 5271 alunos, no qual o com os testes \citeonline{anderson1954test}, \citeonline{shapiro1965analysis}
e \citeonline{shapiro1972approximate}, onde se concluiu que as distribuições das notas são normais. Este resultado será utilizado para atender
os testes paramétricos que pressupõem normalidade dos dados. Os testes a serem aplicados serão o teste ANOVA dado por \citeonline{fisher1928general} para 
avaliar os fatores com mais de 2 categorias e o teste t-Student para as comparações dois a dois \cite{o1908student}.
Estes testes utilizam das médias ($\mu$) de cada grupo para avaliar as distâncias significativas entre eles, no qual previamente avalia se as
variâncias ($\sigma^2$) destes são iguais (Teste B), através do teste proposto por \citeonline{bartlett1954note}.

% Relacao com afazeres
Quanto ao tempo gasto com afazeres domésticos, as variáveis a serem relacionadas são a raça/cor, a escolaridade da mãe
e o sexo, para observar se há indícios de diferenças sociais entre a disponibilidade de tempo em casa para outras possíveis 
tarefas na formação do aprendizado básico. Para estas relações, os testes estatísticos não paramétricos são apropriados,
no qual utiliza-se o teste de \citeonline{kruskal1952use} para fatores com mais de duas categorias (Teste K), e o
teste de \citeonline{mann1947test} para a comparação dois a dois das categorias (Teste W). Estes testes avaliam as
distribuições das informações com base na posição, para verificar se as distâncias entre as categorias são significativas.


% Avaliacao dos testes
Para a avaliar os resultados dos testes, foi proposto o uso da correção de \citeonline{bonferroni1936teoria}, utilizada para os 
testes com mais de duas categorias. Se houver a evidências para rejeitar igualdade destas, a comparação dois a dois é efetuada e a mesma correção
é utilizada. Esta correção é sobre o p-valor, que é avaliado em uma escala de significância, adotado por este estudo
como uma confiança de 95\%, no que diz sobre aceitar a hipótese nula $H_0$.