%----------------------------------------------------------------------------------------------------------------
% File : exemplo.tex
%----------------------------------------------------------------------------------------------------------------

% ---
% Este capítulo, utilizado por diferentes exemplos do abnTeX2, ilustra o uso de
% comandos do abnTeX2 e de LaTeX.


\chapter{Metodologia}

% Avaliação
Este documento tem como objetivo analisar fatores sociais de alunos do 9º ano de 2017 do Brasil,
no qual à pressupostos de possibilidade de influênciar no aprendizado básico destes. O banco de dados
disponibilizado pelo SAEB de 2017 divulgado pelo \citeonline{Saeb2017a)}, possui variáveis com respeito
aos alunos e suas notas em Matemática e Língua Portuguesa, atráves da Prova Brasil de 2017, do censo de 
escolas públicas e amostras de escolas particulares, entre outras variaveis com respeito a escola e fatores
sociais destes.

% Amostra
A análise parte de uma amostragem aleatória simples de 5271 alunos do 9º deste banco de dados, e relaciona 
fatores como raça/cor e sexo dos alunos e localizações das escolas e escolaridade da mãe destes, com as variaveis
a serem explicadas como a soma destas notas e o tempo de afazes domésticos realizados por dia com base nos alunos.

% Relacao com afazeres
O estudo relaciona sobre os alunos a raça/cor, a escolaridade da mãe e o sexo com o tempo de afazeres domésticos, 
para observar se há indícios de diferenças sociais entre a disponibilidade de tempo em casa para outras possiveis 
tarefas na formação do aprendizado básico. Para estas relações, os testes estatisticos não paramêtricos são apropriados,
no qual utiliza-se o teste de \citeonline{kruskal1952use} para fatores com mais de duas categorias (Teste K), e o
teste de \citeonline{mann1947test} para a comparação dois a dois das categorias (Teste W). Estes testes avaliam as
distribuições das informações com base na posição, para comparar as distâncias significativas entre as categorias.

% Relacao com notas
O estudo também relaciona a raça/cor, a escolaridade da mãe e as localizações das escolas com a soma das notas em Matemática 
e Língua Portuguesa com base nos alunos, no intuito de avaliar a influência destas variaveis explicativas sobre o desempenho
total na Prova Brasil. Para avaliar estas relações, um estudo prévio foi realizado com amostras de tamanhos 30, 50 e 100 dos 5271
alunos, no qual o com os testes \citeonline{anderson1954test}, \citeonline{shapiro1965analysis} e \citeonline{shapiro1972approximate}
foi possivel afirmar que a distribuições das notas são normais e os resultados destes não seram abordados neste documento. Os testes 
apropriados para avaliar estas relações, são testes paramétricos, no qual aplica o teste ANOVA dado por \citeonline{fisher1928general} para 
avaliar os fatores com mais de 2 categorias e o teste T para as comparações dois a dois \cite{o1908student}.
Estes testes utilizam das médias de cada grupo para avaliar as distâncias significativas entre eles, no qual previamente avalia se as
variâncias destes são iguais (Teste B), através do teste proposto por \citeonline{bartlett1954note}.

% Avaliacao dos testes
Para a avaliar os resultados dos testes, foi proposto o uso da correção de \citeonline{bonferroni1936teoria}, no qual utiliza-se para os 
testes com mais de duas categorias. Se houver a evidência de rejeitar igualdade destas, a comparação dois a dois é efetuada e a mesma correção
é utiliza. Esta correção é sobre o P-valor, cujo o valor é avaliado em uma escala de significância, adotado por este estudo
como uma confiança de 95\%, no que diz sobre aceitar a hipótese $H_0$.