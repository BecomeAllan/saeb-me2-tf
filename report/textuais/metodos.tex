%----------------------------------------------------------------------------------------------------------------
% File : exemplo.tex
%----------------------------------------------------------------------------------------------------------------

% ---
% Este capítulo, utilizado por diferentes exemplos do abnTeX2, ilustra o uso de
% comandos do abnTeX2 e de LaTeX.

\chapter{Metodologia}
% Relacao com notas
Como variáveis explicativas para a soma das notas, o estudo relaciona o sexo do aluno, a raça/cor, a escolaridade da mãe e as localizações das escolas,
e avalia o quanto essas variáveis influenciam no desempenho na Prova Brasil. Para avaliar estas relações, um estudo prévio foi realizado
com amostras de tamanhos 30, 50 e 100 dos 5.271 alunos em que, com os testes \citeonline{anderson1954test}, \citeonline{shapiro1965analysis}
e \citeonline{shapiro1972approximate}, se concluiu que as distribuições das notas são normais. Este resultado será utilizado para atender
os testes paramétricos que pressupõem normalidade dos dados. Os testes a serem aplicados serão o teste ANOVA Fisher por \citeonline{fisher1928general} para 
avaliar os fatores com mais de 2 categorias e o teste t-Student para as comparações dois a dois \cite{o1908student}.
Estes testes utilizam das médias ($\bar{x}$) de cada grupo para avaliar se as distâncias são significativas entre eles, no qual previamente avalia se as
variâncias ($S^2$) destes são iguais (Teste B), através do teste proposto por \citeonline{bartlett1954note}.

% Relacao com afazeres
Quanto ao tempo gasto com afazeres domésticos, estas mesmas variáveis são exploradas, com exceção das localizações das escolas,
para observar se há indícios de diferenças sociais entre a disponibilidade de tempo em casa para outras possíveis 
tarefas na formação do aprendizado básico. Para estas relações, os testes estatísticos não paramétricos são apropriados,
e utiliza-se o teste de \citeonline{kruskal1952use} para fatores com mais de duas categorias (Teste K), e o
teste de \citeonline{mann1947test} para a comparação dois a dois das categorias (Teste W). Estes testes avaliam as
distribuições das informações com base na posição, para verificar se as distâncias entre as categorias são significativas.

% Avaliacao dos testes
Para avaliar os resultados dos testes, foi proposto o uso da correção de \citeonline{bonferroni1936teoria} para os 
testes com mais de duas categorias. Se houver evidências para rejeitar igualdade destas, a comparação dois a dois é
efetuada e a correção é utilizada. Esta correção é sobre o p-valor, que é avaliado em uma escala de significância,
adotado por este estudo como uma confiança de 95\%, no que diz respeito a aceitar a hipótese nula $H_0$.