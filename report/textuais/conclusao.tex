%----------------------------------------------------------------------------------------------------------------
% File : conclusao.tex
%----------------------------------------------------------------------------------------------------------------

% ---
% Conclusão
\chapter{Conclusão}

De acordo com os resultados obtidos utilizando testes estatísticos com uma confiança de 95\%, 
houve indícios significativos que para alguns fatores sociais,
há desigualdades entre as avaliações das relações neste estudo com a soma das notas em Língua
Portuguesa e Matemática ou o tempo de afazeres diários em casa por parte dos alunos do 9\textsuperscript{o} ano
do ensino fundamental.

A raça/cor não diz sobre os tempos destes afazeres, mas ao analisar os resultados destes testes sobre a soma destas notas,
os alunos que se autodeclaram com raça/cor Branca, em média, obtiveram maiores notas.

Sobre o sexo do aluno, aqueles que são do sexo feminino acabam por fazer mais destes afazeres, não havendo
o mesmo tempo disponível em casa hipoteticamente que comparado ao outro sexo. Já para o desempenho médio entre os sexos,
não houve evidência de desigualdade entre esta soma. 

O nível de escolaridade da mãe do aluno, diz sobre o período de tempo gasto por ele nestas tarefas diárias,
no qual apenas aquelas mães que o aluno não sabe a escolaridade, resultam em exercer menos tempo nestas tarefas.
Com base nesta soma das notas, estes alunos que não sabem, obtém a média equivalente as mães que tem o 5\textsuperscript{o}
incompleto ou as que tem o 9\textsuperscript{o} completo sobre o ensino fundamental, mas de forma geral,
o nível de escolaridade da mãe, diz em média, que quanto maior este nível, maior será a nota.

Para a localização da escola do aluno, foi analisado apenas esta soma das notas, no qual houve evidência
substâncial para afirmar que aqueles que estudam em uma zona urbana, acabam por ter notas superiores que aquelas
que estudam em uma zona rural. 

De forma geral, fatores sociais influenciam o alunos do 9\textsuperscript{o} no desenvolvimento da educação básica,
pelo qual existe desigualdades nestes fatores que prejudicam a igualdade da formação do conhecimento destas turmas,
ao passo que alguns alunos dispõe de mais tempo em casa ou exemplos maternos, acabam por ter influência no aluno sobre as 
notas avaliadas pela Prova Brasil efetuada de forma igualitária para toda estas turmas.

% TABELA DE ESC_MAE TAVA TROCADAAAAA !!!!!!!!!!!!!!!!!!!!!!!!!
%\begin{enumerate}[label=\alph*)]

%\item A variável Raça/Cor não exerce influência no tempo de afazeres domésticos.
%\item A variável Escolaridade da Mãe exerce influência no tempo de afazeres domésticos.
%\item A variável Sexo exerce influência no tempo de afazeres domésticos.
%\item A proporção de alunos que gastam tempo com afazeres domésticos, cujo as mães que nunca estudaram se difere com todos os outros níveis escolares que, em grande parte gastam menos de uma hora.
%\item O mesmo se observa para as mães com o 5º ano incompleto, no qual o único grau que não se difere é dos alunos que responderam que não sabiam sobre o nível educacional da mãe.
%\item Não existe diferença sobre as médias das notas dos estudantes de acordo com o sexo.
%\item A média das notas dos estudantes varia de acordo com a Raça/Cor e com o nível de escolaridade da mãe.
%\item Há uma clara desigualdade entre as notas por raça quando se compara a branca com as demais, e quase todas as desigualdades derivaram dela.
%\item As únicas desigualdades que não foram entre comparações de indivíduos de cor branca com outros foi a comparação entre indivíduos pardos e pretos, e também acomparação entre indivíduos pardos e aqueles que não quiseram declarar.
%\item Somente aqueles que declararam não saber a escolaridade da mãe tiveram suas hipóteses de igualdade das médias rejeitadas, indicando diferença entre as notas médias desses estudantes com aqueles cujas mães completaram o EnsinoFundamental I e II, ensino médio e faculdade. A única hipótese de igualdade das notas desses estudantes aceita foi a com estudantes cujas mães nunca haviam estudado.

%\end{enumerate}