%----------------------------------------------------------------------------------------------------------------
% File : conclusao.tex
%----------------------------------------------------------------------------------------------------------------

% ---
% Conclusão
\chapter{Conclusão}

De acordo com os resultados do estudo, há fatores que comprovam a diferença de qualidade do ensino básico para as turmas do 9º ano de 2017 do Brasil. Sobre as notas em Língua Portuguesa, o fator de tempo de uso de telas por parte dos alunos, há diferenças substanciais para poder afirmar com 95\% de confiança, que os alunos que usam menos de 1 hora ou não usam, obtiveram notas menores que comparado com os demais que usa as telas por mais tempo. As notas em Matemática relacionada a região das escolas dos alunos, com a mesma confiança, observou-se que alunos da regiões Sul e Centro-oeste detém de maiores notas que comparados com as regiões Norte e Nordeste, havendo evidências de desigualdade regional por parte do ensino básico em Matemática para o 9º ano, com base nas associações com as notas avaliativas realizados pelos alunos na prova Brasil em 2017.