%----------------------------------------------------------------------------------------------------------------
% File : introducao.tex
%----------------------------------------------------------------------------------------------------------------

% Introdução (exemplo de capítulo sem numeração, mas presente no Sumário)
\chapter*[Introdução]{Introdução}
\addcontentsline{toc}{chapter}{Introdução}

% Avaliação
Este documento tem como objetivo analisar os fatores sociais de alunos brasileiros do 9º ano de 2017,
sobre os quais avalia possibilidades de diferenças no desempenho do aprendizado básico destes.
Para a análise, serão utilizados dados do Sistema de Avaliação da Educação Básica (SAEB) de 2017 
divulgado pelo \citeonline{Saeb2017a)}. O banco de dados utilizado no estudo é constituído por dados de
estudantes do 9\textsuperscript{o} ano que realizaram a Prova Brasil no ano de 2017.

% Saeb
O SAEB avalia o desempenho em Matemática e Língua Portuguesa por meio de testes e questionários sobre fatores sociais de
alunos, aplicados bienalmente em redes públicas
e amostras de escolas privadas sobre turmas do 5º e 9º anos do ensino fundamental e 
3º ano do ensino médio.

% Amostra
Esta análise parte de uma amostragem aleatória simples de 5.271 alunos deste banco de dados, e relaciona 
fatores como raça/cor, sexo dos alunos, localizações das escolas e escolaridade da mãe destes, com as variáveis
a serem explicadas como a soma destas notas e o tempo de afazeres domésticos realizados por dia com base nos alunos.

% Banco de dados

A aplicação dos testes de hipóteses para a análise da amostra, bem como dos gráficos (para a análise descritiva), utiliza ferramentas
do software \textsc{R} nas versões 3.6.3 e 4.0.3. Parte do processamento dos dados foi feito com o software \textsc{Python} versão 3.7.3.\footnote{Pacotes externos usados para a manipulação dos dados:\\\textit{R: tidyverse, data.table, reshape2, patchwork, EnvStats, PMCMR; \\Python: pandas}}

