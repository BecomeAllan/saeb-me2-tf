 %----------------------------------------------------------------------------------------------------------------
% File : resumo.tex
%----------------------------------------------------------------------------------------------------------------

% resumo na língua vernácula (obrigatório)
\setlength{\absparsep}{18pt} % ajusta o espaçamento dos parágrafos do resumo
\begin{resumo}  
    

  A fim de compreender o comportamento do tempo de afazeres domésticos e o desempenho de
  estudantes que realizaram a Prova Brasil, foi realizado uma série de testes estatísticos para saber como 
  outras variáveis demográficas influenciam nesses fatores. Foram utilizadas como variáveis explicativas: Raça/Cor,
  Escolaridade da Mãe, Sexo e Localização.
 
  Diante disso, foi possível observar uma diferença significativa entre o tempo de afazeres domésticos de acordo com o sexo,
  em que estudantes do sexo feminino realizam mais tempo de atividades domiciliar. A escolaridade da mãe também foi um fator existente
  para decretar uma diferença no tempo de atividades domésticas, já que estudantes que não sabem o nível de escolaridade da mãe 
  trabalham menos em casa do que os outros grupos.
 
  Como fatores responsáveis pelo desempenho, foi observada a diferença entre estudantes de fenótipo branco com o restante
  das raças/cores, em que esse grupo demonstrou um desempenho maior nas provas. O nível de Escolaridade da Mãe também indica 
  certa tendência a aumento nas notas de acordo com a o grau de formação da figura materna dos estudantes. O último 
  elemento entre os estudados que reflete nas notas é a localização, em que estudantes da região urbana tiveram desempenho superior 
  aos da zona rural.
    
  \noindent
  \textbf{Palavras-chaves}: \kwords
\end{resumo}
